\section*{Casos de Uso}

\subsection*{Listado de Casos de Uso}
\begin{description}
\item[CU-01] Autentic\'andose al sistema
\item[CU-02] Cargando oferta a trav\'es de la Web-API
\item[CU-03] Consultando oferta a trav\'es de API-Internet
\item[CU-04] Consultando oferta a trav\'es de p\'agina web TPA
\item[CU-05] Enviando sugerencia sobre el sistema
\item[CU-06] Configurando oferta sugerida %(24hs)
\item[CU-07] Buscando informaci\'on de oferta sugerida %(32hs)
\item[CU-08] Generando reporte de ofertas dudosas con Spam-Buster %(64hs)
\item[CU-09] Invalidando oferta %(56hs)
\item[CU-10] Cargando/Consultando oferta por P\'agina Web/Red Social 
\item[CU-11] Generando reporte de ofertas dudosas con M\'odulo Propio %(64hs)
\item[CU-12] Registrar informaci\'on de uso del sistema %(24)
\item[CU-13] Asignando confiabilidad a oferta %(48)
\item[CU-14] Asignando confiabilidad a usuario %(40)
\item[CU-15] Comparando SpamBuster con M\'odulo de ofertas dudosas %(160hs)
\item[CU-16] Configurando sistema de confianza personal %(136hs)
\item[CU-17] Mostrando mapa oferta %(104hs)
\item[CU-18] Analizando web en busca de ofertas %(48hs)
\item[CU-19] Cargando/Consultando oferta por SMS %(36hs)
\item[CU-20] Consultando informaci\'on estad\'istica del sistema %(80 hs)
\item[CU-21] Configurando sistema de ofertas
\end{description}

\subsection*{Detalle de los Casos de Uso}

En esta secci\'on incluiremos una lista de los casos de uso identificados para el sistema a implementar durante la primera iteraci\'on, con una breve descripci\'on de alto nivel para cada uno. Se trata solamente de interacciones entre el sistema y agentes externos (es decir, el usuario y otros sistemas). Por esta razon, esta clasificaci\'on no contiene absolutamente todo el trabajo a realizar.  

En particular, no se describe aqu\'i el trabajo requerido para permitir a nuestro sistema detectar las ofertas posteadas en las diversas redes soportadas. 

\textbf{CU-01: Autentic\'andose al sistema} Un usuario con una cuenta de usuario v\'alida podr\'a utilizarla para autenticarse con el sistema. Est\'a cuenta podr\'a ser una cuenta OpenId, de alguno de los sistemas externos soportados o de usuario interno para tareas de configuraci\'on y administraci\'on

\textbf{CU-02: Cargando oferta a trav\'es de la Web-API} Un usuario podr\'a cargar una nueva oferta de un producto v\'alido en el sistema. En caso de que el usuario est\'e aut\'enticado se le dar\'a prioridad de acuerdo a . Los usuarios pagos podr\'an adem\'as incorporar m\'as informaci\'on a la oferta para que esta se ubique como un aviso esponsoreado.

\textbf{CU-03: Consultando oferta a traves de API-Internet} Un usuario podr\'a realizar consultas sobre las mejores ofertas que tiene el sistema. Los resultados de b\'usqueda estar\'an ordenados por las ofertas de mayor importancia, esto definido por las reglas internas.

\textbf{CU-04: Consultando oferta a trav\'es de p\'agina web TPA} Un usuario podr\'a realizar consultas sobre las mejores ofertas que tiene el sistema. Los resultados de b\'usqueda estar\'an ordenados por las ofertas de mayor importancia, esto definido por las reglas internas. En caso de lo usuarios aut\'enticados, de tenerlo definido, se utilizar\'an adem\'as sus preferencias de confianza en las fuentes (ej: ciertos usuarios, webs).

\textbf{CU-05: Enviando sugerencia sobre el sistema} Un usuario del sistema, sin importar si est\'a autenticado o no, podr\'a enviar sugerencias sobre el sistema. Las sugerencias Spam, de haberlas, deber\'an ser tratadas con una estrategia anti-spam.
