\section*{Conclusi\'on}

Al momento de comparar las metodolog\'ias utilizadas para la primera y segunda parte, notamos c\'omo \'estas se ajustan a las caracter\'isticas de cada uno de los escenarios: tiempo de planeamiento, incorporaci\'on de cambios, dependencias entre los distintos hitos, los entregables esperables en cada etapa, cuan involucrado debe estar el cliente en el proceso de desarrollo. 

Puede verse que las m\'etodolog\'ias hacen foco en distintas cosas, Scrum hace foco en ir explorando el problema a medida que se lo va resolviendo y en los pedidos del momento del cliente. RUP por otro lado hace incapi\'e en el an\'alisis y planificaci\'on antes que otras tareas para resolver el problema.

Esto hace que en scrum haya una mayor visibilidad del avance hacia el cliente a fin de lograr su satisfacci\'on y adaptabilidad a los cambios en el negocio o necesidades del cliente, esto \'ultimo puede verse con el cambio de los stories dentro del backlog y en la selecci\'on de las mismas antes de comenzar un sprint. Por otra parte la manera que tiene esta metodolog\'ia agil para afrontar los problemas que pueden surgir es de alguna manera mediante las standup meetings diarias. 

En RUP lo que logra el foco en el an\'alisis y planificaci\'on es que pueda mitigarse (o al menos conocerse) que inconvenientes pudieran surgir, as\'i como ver la viabilidad de los requerimientos del cliente y en que cuestiones debe hacerse foco para satisfacer al mismo, esto \'ultimo utilizando los escenarios de calidad. Dado que la metodolog\'ia permite planificar el proyecto por entero es posible de alguna manera estimar su costo, lo cual en organizaciones que manejan prespuestos anuales puede ser sumamente \'util a la hora de seleccionar proyectos.

Entendemos que estas dos metodolog\'ias si bien pueden considerarse como \"rivales\", podr\'ian verse como complementarias. RUP podr\'ia utilizarse para el an\'alisis macro de un proyecto, para poder entender el problema y diseñar su arquitectura, y scrum para en el desarrollo de los distintos módulos que componen la arquitectura generada, benefici\'andose as\'i el proyecto de la adaptabilidad a los cambios que pudieran surgir a posteriori en el negocio.

