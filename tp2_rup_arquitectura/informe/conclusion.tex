\section*{Conclusi\'on}

Al momento de comparar las metodolog\'ias utilizadas para la primera y segunda parte, notamos c\'omo \'estas se ajustan a las caracter\'isticas de cada uno de los escenarios: tiempo de planeamiento, velocidad de incorporaci\'on de cambios, dependencias entre los distintos hitos, criticidad de los entregables (calidad y responsabilidades).

En el caso de la primera parte, hay varios puntos importantes a tener en cuenta: el entregable es un prototipo, se incorporan nuevas tecnolog\'ias que deben ser investigadas, se desea tener la funcionalidad en un tiempo acotado, puede haber dependencias entre los componentes, pero son desconocidas de antemano. En este contexto, utilizar una metodolog\'ia \'agil como Scrum nos permite adaptarnos mejor a estos puntos, ya que, si bien hay una identificaci\'on inicial de los requerimientos y casos de uso, no nos exige tener un plan detallado de dependencias, y se pueden agregar tareas nuevas (y quitar otras) de acuerdo a la necesidad, manteniendo la duraci\'on de la iteraci\'on como la \'unica constante. Adem\'as la flexibilidad en la planificaci\'on inicial es complementada con los tiempos cortos de reacci\'on, producto de las pr\'acticas como la reuni\'on diaria standup.

En la segunda parte, hay otros puntos que favorecen el uso de pr\'acticas como RUP: la longitud del proyecto es mucho mayor, los entregables son para uso productivo (mayor requisito de calidad y m\'as responsabilidades), y entre estos \'ultimos la arquitectura, que requiere un tiempo de dise\~no superior, y que con certeza define un \'arbol de dependencias mucho m\'as marcado. Al utilizar metodolog\'ias como RUP se tiene un control mayor de los tiempos totales del proyecto, y los recursos involucrados; adem\'as se resuelven dependencias y factores de riesgo que pueden tener un impacto mucho mayor en la totalidad del proyecto.