\subsection{Arquitectura} % (fold)
\label{sub:Arquitectura}

Durante el an\'alisis generamos diversas User Stories de investigaci\'on, estas nos permitieron hacer un an\'alisis de la arquitectura antes de comenzar a implementarla.
Con pruebas de concepto que nos permitieron entender la factibilidad de dicha implementaci\'on.

Los resultados de dicho an\'alisis se encuentra al final de esta secci\'on, en los siguientes parrafos haremos un an\'alisis de la tecnolog\'ia elegida.

\subsection{Lenguaje}
La decisi\'on de utilizar un lenguaje de tipado din\'amico fue t\'acita, quisimos evitar las restricciones sem\'anticas que nos impone un lenguaje de tipado est\'atico.  
Tambi\'en buscamos un lenguaje de Objetos Puro, dejando Python afuera de esta decisi\'on.  
Finalmente optamos por Ruby, din\'amicamente tipado, de objetos puro, con closures.


No es casualidad que la elecci\'on del lenguaje nos afectara tambi\'en en la elecci\'on de las tecnolog\'ias de soporte de aplicaci\'on:
\begin{itemize}
    \item Servidor SQL, para el soporte de datos de backend
    \item RESTFul server, para la interfaz de aplicaci\'on
    \item Engine de templating para las vistas
\end{itemize}

\subsection{Servidor SQL}
Utilizamos SQLite 3 por su portabilidad y facilidad de instalaci\'on.

\subsection{Server}

Para brindar los servicios al usuario, utilizamos HTML con una api HTTPRest. Nuestro servidor es Sinatra. De f\'acil utilizaci\'on solo es necesario conectar los servicios HTTP a la capa del controlador.

Esto nos permite desacoplar la aplicaci\'on de la interfaz de usuario, en la versi\'on actual entregamos una interfaz modesta en HAML, un lenguaje de templating cuya transormaci\'on a HTML es nativa en Ruby + Sinatra.

\subsection{Modelo}

Para el modelo de aplicaci\'on aplicamos una versi\'on reducida de MVC.

Por un lado tenemos el servidor de aplicaci\'on (sinatra como comentabamos en la secci\'on anterior) y por otro lado tenemos un controller cuyas responsabilidades se ven acotadas a la validaci\'on del input de usuario y la delegaci\'on del procesamiento a ciertas acciones definidas en la capa de servicios.

Los servicios que actualmente soporta el controlador son los de b\'usqueda.

\subsection{Extensibilidad}

Nuestro dise\~no como veremos (y justificaremos) en las siguientes secciones, es extensible en:

\begin{itemize}
    \item Tipos de filtros: nuevos tipos de b\'usquedas podr\'an agregarse sin modificar ni la validaci\'on de los par\'ametros ni la interfaz de usuario. Esto lo conseguimos mediante una sencilla t\'ecnica de metaprogramaci\'on.
    \item Tipos de consumo de datos de Twitter: actualmente soportamos la api on-line y la api off-line. 
    \item Estrat\'egias de parsing de twitts: en esta versi\'on acotada utilizamos una extracci\'on de datos posicional por medio de regular expressions, esto podr\'a ser modificado para utilizar un interprete de lenguaje natural que extraiga estos datos cuya ambig\"uedad es tan variada.
    \item Interfaces de usuario: actualmente brindamos una interfaz HTML pero podr\'ia interactuar con una UI distinta, braile, de escritorio, distribuida, etc.
\end{itemize}

% subsection Arquitectura (end)

