\subsection{Retropesctiva}
Principalmente disfrutamos de aprender Ruby, siempre es bueno aprender un lenguaje nuevo.

Creemos que las dificultados no estuvieron en el aprendizaje de nuevas herramientas, si no más bien en tratar de mantener el ritmo de trabajo.

Mención aparte merece el tema del trabajo en equipo. Acá nos encontramos con algunas dificultades, muchas por desconocimiento de cómo trabaja el otro. Ya que salvo Daniel y Hernan el resto nunca había trabajado en un TP juntos. Algunos miembros se adaptaron al dinámica de grupo que se estableció y otros no. 

Esto creemos que afectó en la entrega dado que en la recta final algunos miembros del grupo se sobrecargaron de tareas.

También hicimos mucho foco en las herramientas y la implementación y no tanto en el diseño formal. Nos dejamos llevar por el entusiasmo por lo nuevo, si bien estamos convencidos de que el diseño que pensamos es bueno.

Este foco por la implementación y el ritmo tardío de trabajo hizo que nos comunicáramos poco con el product owner para validar nuestro avance. Tal vez acá nos podría haber llamado un poco más la atención, como entendemos que hubiera sucedido en un escenario real (si bien sabemos que ya estamos grandecitos).

En siguientes sprints, entendemos que deberíamos mejorar el ritmo de trabajo, tratar de mantener un ritmo de trabajo más constante. Esto nos ayudará a la velocidad del equipo, que creemos que estuvo bastante cercana a lo esperado salvo que, como comentamos más arriba, requirió cierto sobreesfuerzo. Por otra parte una mayor comunicación con el product owner entendemos producirían entregas más acorde a lo que espera.


