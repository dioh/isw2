%\documentclass[mathserif, blue]{beamer}%Agregar ,notes antes del ] para incluir las notas.
\documentclass{beamer}
% Paquetes usados
\usepackage[spanish]{babel}
\usepackage[utf8]{inputenc}
\usepackage{indentfirst}
\usepackage{beamerthemeshadow}
\usepackage{xspace}
\usepackage{latexsym}

%\beamertemplateshadingbackground{green!5}{yellow!10}
%\beamertemplatetransparentcovereddynamic
%\setbeamertemplate{blocks}[rounded][shadow=false] 
\usetheme{Warsaw}
%\usetheme{Berlin}[shadow=false]
%setbeamercovered{dynamic}

\parskip=1ex

\title{Papers Fundacionales}
\subtitle{7 Mitos de los Métodos Formales}
\author{Foguelman - Modrow - Tilli}
\institute{DC - UBA}

\begin{document}

\frame{\titlepage}

\section{Mitos}
\subsection{Mito 1}
\begin{frame}
Los m\'etodos formales pueden garantizar que el software es perfecto 
%\begin{itemize}[<+->]
%\item En una clase pr\'actica donde se vean temas de seguridad en redes, se puede dar como ejercicio introductorio a firewalls
%\end{itemize}
\end{frame}

\subsection{Mito 2}
\begin{frame}
(Los m\'etodos formales) funcionan probando que los programas son correctos
\end{frame}

\subsection{Mito 3}
\begin{frame}
S\'olo los sistemas altamente cr\'iticos se benefician de los m\'etodos formales
\end{frame}

\subsection{Mito 4}
\begin{frame}
(Los m\'etodos formales) involucran matem\'atica compleja
\end{frame}

\subsection{Mito 5}
\begin{frame}
(Los m\'etodos formales) incrementan los costos de desarrollo
\end{frame}

\subsection{Mito 6}
\begin{frame}
(Los m\'etodos formales) son incomprensibles para los clientes
\end{frame}

\subsection{Mito 7}
\begin{frame}
(Los m\'etodos formales) nadie los usa para proyectos reales
\end{frame}

\end{document}
