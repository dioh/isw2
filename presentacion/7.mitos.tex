%\documentclass[mathserif, blue]{beamer}%Agregar ,notes antes del ] para incluir las notas.
\documentclass{beamer}
% Paquetes usados
\usepackage[spanish]{babel}
\usepackage[utf8]{inputenc}
\usepackage{indentfirst}
\usepackage{beamerthemeshadow}
\usepackage{xspace}
\usepackage{latexsym}

\setlength{\paperwidth}{4.75in}
\setlength{\paperheight}{3.4in}
\setlength{\textwidth}{3.96in}
\setlength{\textheight}{3.4in}
%\beamertemplateshadingbackground{green!5}{yellow!10}
%\beamertemplatetransparentcovereddynamic
%\setbeamertemplate{blocks}[rounded][shadow=false] 
\usetheme{Warsaw}
%\usetheme{Berlin}[shadow=false]
%setbeamercovered{dynamic}

\parskip=1ex

\title{Papers Fundacionales}
\subtitle{7 Mitos acerca de los M\'etodos Formales}
\author{Foguelman - Modrow - Tilli}
\institute{DC - UBA}

\begin{document}

\frame{\titlepage}
\section{Introducci\'on}
\begin{frame}{Introducci\'on}

\end{frame}

\section{Mitos}
\begin{frame}{Mito 1: Pueden garantizar que el software es perfecto }
\begin{itemize}[<+->]
\item[-] L\'imite en las demostraciones (mundo real vs modelo)
\item[-] Error en la especificaci\'on
\item[+] Demostraci\'on de correctitud: propiedades globales, y relaci\'on entre programa y especificaci\'on
\item[+] Encontrar errores
\item \textbf{Hecho:} Los m\'etodos formales son muy \'utiles para encontrar errores de manera temprana y pueden eliminar casi completamente algunas clases de errores
\end{itemize}
\end{frame}

\begin{frame}{Mito 2: Son \'utiles s\'olo para probar correctitud}
\begin{itemize}[<+->]
\item \textbf{Hecho:} Trabajan generalmente haciendo que pienses mucho sobre el sistema que pretendes construir
\end{itemize}
\end{frame}
 
\begin{frame}{Mito 3: S\'olo se benefician los sistemas altamente cr\'iticos} 
\begin{itemize}[<+->]
\item \textbf{Hecho:} Son \'utiles para casi cualquier tipo de aplicaci\'on
\end{itemize}
\end{frame}
 
\begin{frame}{Mito 4: Involucran matem\'atica compleja}
\begin{itemize}[<+->]
\item \textbf{Hecho:} Se basan en especifaciones matem\'aticas que son m\'as f\'acil de entender que un programa
\end{itemize}
\end{frame}
 
\begin{frame}{Mito 5: Incrementan los costos de desarrollo}
\begin{itemize}[<+->]
\item \textbf{Hecho:} Pueden decrementar los costos de desarrollo
\end{itemize}
\end{frame}
 
\begin{frame}{Mito 6: Son incomprensibles para los clientes}
\begin{itemize}[<+->]
\item \textbf{Hecho:} Pueden ayudar a los clientes a entender que est\'an comprando
\end{itemize}
\end{frame}
 
\begin{frame}{Mito 7: Nadie los usa para proyectos reales}
\begin{itemize}[<+->]
\item \textbf{Hecho:} Son utilizados con \'exito proyectos \'utiles de la industria
\end{itemize}
\end{frame}

\end{document}
