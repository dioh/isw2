%\documentclass[mathserif, blue]{beamer}%Agregar ,notes antes del ] para incluir las notas.
\documentclass{beamer}
% Paquetes usados
\usepackage[spanish]{babel}
\usepackage[utf8]{inputenc}
\usepackage{indentfirst}
\usepackage{beamerthemeshadow}
\usepackage{xspace}
\usepackage{latexsym}

\setlength{\paperwidth}{4.75in}
\setlength{\paperheight}{3.4in}
\setlength{\textwidth}{3.96in}
\setlength{\textheight}{3.4in}
%\beamertemplateshadingbackground{green!5}{yellow!10}
%\beamertemplatetransparentcovereddynamic
%\setbeamertemplate{blocks}[rounded][shadow=false] 
\usetheme{Warsaw}
%\usetheme{Berlin}[shadow=false]
%setbeamercovered{dynamic}

\parskip=1ex

\title{Papers Fundacionales}
\subtitle{7 Mitos acerca de los M\'etodos Formales}
\author{Foguelman - Modrow - Tilli}
\institute{DC - UBA}

\begin{document}

\frame{\titlepage}
\section{Introducci\'on}
\begin{frame}{Introducci\'on}
\begin{itemize}
\item Art\'iculo aparece en IEEE Software, September 1990
\item Anthony Hall 
\begin{itemize}
\item ingeniero de software y sistemas
\item especializado en m\'etodos formales
\item qu\'imico
\end{itemize}
\item Trabajaba en proyecto CASE (herramienta para SSADM) al momento del art\'iculo
\end{itemize}

\end{frame}

\section{Mitos}
\begin{frame}{Mito 1: Pueden garantizar que el software es perfecto }
\begin{itemize}[<+->]
\item[-] Las demostraciones son falibles 
\item[-] L\'imite en las demostraciones (mundo real vs modelo)
\item[-] Error en la especificaci\'on
\item[+] Demostraci\'on de correctitud: propiedades globales, y relaci\'on entre programa y especificaci\'on
\item[+] Exponer errores
\end{itemize}
\uncover<6->{\textbf{Hecho:} Los m\'etodos formales son muy \'utiles para exponer errores de manera temprana y pueden eliminar casi completamente algunas clases de errores}
\end{frame}

\begin{frame}{Mito 2: Son \'utiles s\'olo para probar correctitud}
\begin{itemize}[<+->]
\item[-] Prueba de correctitud dificultosa a pesar del uso de metodos formales
\item[+] Ayuda a la clarificaci\'on de requerimientos
\item[+] Construcci\'on de la especificaci\'on y detecci\'on temprana de errores
\item[+] Demostraci\'on de propiedades de la especificaci\'on
\item[+] Implementaci\'on semiautom\'atica e iterativa
\end{itemize}
\uncover<5->{\textbf{Hecho:} Trabajan generalmente haciendo que pienses mucho sobre el sistema que pretendes construir}
\end{frame}
 
\begin{frame}{Mito 3: S\'olo se benefician los sistemas altamente cr\'iticos}
\begin{itemize}[<+->]
\item[+] Se benefician muchos sistemas: cr\'iticos, replicados, embebidos en hardware, de alta calidad
\item[+] Objetividad, mantenimiento, facilidad de construcci\'on, visibilidad 
\item[+] Monitoreo del desarrollo
\item[+] Distinta granularidad en la formalidad
\end{itemize}
\uncover<5->{\textbf{Hecho:} Son \'utiles para casi cualquier tipo de aplicaci\'on}
\end{frame}
 
\begin{frame}{Mito 4: Involucran matem\'atica compleja}
\begin{itemize}[<+->]
\item[-] Dificultad en modelar el mundo real
\item[-] Muy abstracto o muy espec\'ifico 
\item[-] Se necesita un experto, algunas veces un consultor experto
\item[+] Mayormente sólo se necesita l\'ogica y teor\'ia de conjuntos
\item[+] En general epecificaci\'on m\'as corta y sencilla que implementaci\'on
\end{itemize}
\uncover<6->{\textbf{Hecho:} Se basan en especifaciones matem\'aticas que son m\'as f\'aciles de entender que un programa}
\end{frame}
 
\begin{frame}{Mito 5: Incrementan los costos de desarrollo}
\begin{itemize}[<+->]
\item[-] Dificil de medir avance durante especificaci\'on % => HEM: que casos?? => (aunque hay casos de \'exito
\item[-] Las especificaciones nunca son perfectas
\item[?] Cambio en el modelo de ciclo de vida % HEM: No entiendo a que se refiere
\item[+] Evidencia de mejores medidores de productividad % HEM: No me queda clara la frase 
\end{itemize}
\uncover<4->{\textbf{Hecho:} Pueden decrementar los costos de desarrollo}
\end{frame}
 
\begin{frame}{Mito 6: Son incomprensibles para los clientes}
\begin{itemize}[<+->]
\item[+] Inclusi\'on de traducci\'on al lenguaje natural
\item[+] Demostraci\'on de cumplimiento de funcionalidad
\item[+] Uso de prototipos, o animar la especificaci\'on
\end{itemize}
\uncover<4->{\textbf{Hecho:} Pueden ayudar a los clientes a entender qu\'e est\'an comprando}
\end{frame}
 
\begin{frame}{Mito 7: Nadie los usa para proyectos reales}
\begin{itemize}[<+->]
\item[+] Evidencia de uso en varios sistemas y disciplinas: procesamiento de transacciones, hardware, compiladores, herramientas de desarrollo de software, control de reactores
\end{itemize}
\uncover<2->{\textbf{Hecho:} Son utilizados con \'exito en proyectos \'utiles de la industria}
\end{frame}

\begin{frame}

    \begin{itemize}
        \item Ninguna panacea
        \item Aplicables a cualquier aplicaci\'on, detectan ciertos tipos de errores
        \item Pueden achicar costos
        \item Es preciso revisar las metodolog\'ias a utilizar
    \end{itemize}
    
\end{frame}

\begin{frame}
    \begin{center}
        
    \LARGE{?`Preguntas?}
    \end{center}
\end{frame}

\end{document}
