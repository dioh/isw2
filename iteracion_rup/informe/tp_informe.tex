\documentclass[a4paper]{article}
\usepackage[utf8]{inputenc}
\usepackage{listings}
\usepackage{graphicx}
\usepackage{epsfig}
\usepackage{riesgos}
\usepackage[font=small,labelfont=bf]{caption}
\hoffset=-2.5cm
\textwidth=17cm
\parskip=1ex



%opening
\title{Ingeniería de Software II\\ \textbf{Sistema ``Twitteando para Ahorrar''}}
\author{\textbf{Grupo 6}\\ 1º Cuatrimestre 2013} 
\date{}


\begin{document}

\maketitle
\vspace{10cm}
\begin{center}

\begin{tabular}{|c|c|c|}
\hline
\hline
\textbf{LU}&\textbf{Nombre}&\textbf{email}\\
\hline
667/06&Daniel Foguelman &dfoguelman@dc.uba.ar\\
\hline
767/03&Hernán Modrow&hmodrow@gmail.com\\
\hline
511/00&Leonardo Tilli&leotilli@gmail.com\\
\hline
\hline
\end{tabular}
\end{center}
\newpage

\section{Parte I}

 En este informe presentamos la planificación para el proyecto TPA para las etapas subsiguientes a lo creado en la entrega anterior. En este proceso, integraremos el Core de Twiteando para Ahorrar con múltiples servicios (Redes sociales, detección de fraude, etc), tendremos requerimientos funcionales y no-funcionales de media y gran complejidad. Esto Nos mueve del paradigma SCRUM por uno más tradicional, RUP, haciendo preponderancia en la planificación y análisis de requerimientos antes que en la adaptabilidad al cambio y visibilidad del proyecto. \\

 El plan desarrollado para esta nueva etapa sigue el modelo RUP, y se documenta a continuaci\'n. Primero se describir\'an los casos de uso identificados. En base a esto se detallan las distintas iteraciones que se planificaron y de qu\'e consta cada una. En la \'ultima secci\'on se encuentra el an\'alisis de los riesgos m\'as importantes que se detectaron.


\section*{Casos de Uso}

\subsection*{Listado de Casos de Uso}
\begin{description}
\item[CU-01] Autentic\'andose al sistema
\item[CU-02] Cargando oferta a trav\'es de la Web-API
\item[CU-03] Consultando oferta a trav\'es de API-Internet
\item[CU-04] Consultando oferta a trav\'es de p\'agina web TPA
\item[CU-05] Enviando sugerencia sobre el sistema
\item[CU-06] Configurando oferta sugerida %(24hs)
\item[CU-07] Buscando información de oferta sugerida %(32hs)
\item[CU-08] Generando reporte de ofertas dudosas con Spam-Buster %(64hs)
\item[CU-09] Invalidando oferta %(56hs)
\item[CU-10] Cargando/Consultando oferta por P\'agina Web/Red Social 
\item[CU-11] Generando reporte de ofertas dudosas con M\'odulo Propio %(64hs)
\item[CU-12] Registrar información de uso del sistema %(24)
\item[CU-13] Asignando confiabilidad a oferta %(48)
\item[CU-14] Asignando confiabilidad a usuario %(40)
\item[CU-15] Comparando SpamBuster con M\'odulo de ofertas dudosas %(160hs)
\item[CU-16] Configurando sistema de confianza personal %(136hs)
\item[CU-17] Mostrando mapa oferta %(104hs)
\item[CU-18] Analizando web en busca de ofertas %(48hs)
\item[CU-19] Cargando/Consultando oferta por SMS %(36hs)
\item[CU-20] Consultando información estadística del sistema %(80 hs)
\item[CU-21] Configurando sistema de ofertas
\end{description}

\subsection*{Detalle de los Casos de Uso}

En esta sección incluiremos una lista de los casos de uso identificados para el sistema a implementar durante la primera iteración, con una breve descripción de alto nivel para cada uno. Se trata solamente de interacciones entre el sistema y agentes externos (es decir, el usuario y otros sistemas). Por esta razon, esta clasificación no contiene absolutamente todo el trabajo a realizar.  

En particular, no se describe aquí el trabajo requerido para permitir a nuestro sistema detectar las ofertas posteadas en las diversas redes soportadas. 

\textbf{CU-01: Autenticándose al sistema} Un usuario con una cuenta de usuario válida podrá utilizarla para autenticarse con el sistema. Está cuenta podrá ser una cuenta OpenId, de alguno de los sistemas externos soportados o de usuario interno para tareas de configuración y administración

\textbf{CU-02: Cargando oferta a través de la Web-API} Un usuario podrá cargar una nueva oferta de un producto válido en el sistema. En caso de que el usuario esté auténticado se le dará prioridad de acuerdo a . Los usuarios pagos podrán además incorporar más información a la oferta para que esta se ubique como un aviso esponsoreado.

\textbf{CU-03: Consultando oferta a traves de API-Internet} Un usuario podrá realizar consultas sobre las mejores ofertas que tiene el sistema. Los resultados de búsqueda estarán ordenados por las ofertas de mayor importancia, esto definido por las reglas internas.

\textbf{CU-04: Consultando oferta a través de página web TPA} Un usuario podrá realizar consultas sobre las mejores ofertas que tiene el sistema. Los resultados de búsqueda estarán ordenados por las ofertas de mayor importancia, esto definido por las reglas internas. En caso de lo usuarios auténticados, de tenerlo definido, se utilizarán además sus preferencias de confianza en las fuentes (ej: ciertos usuarios, webs).

\textbf{CU-05: Enviando sugerencia sobre el sistema} Un usuario del sistema, sin importar si está autenticado o no, podrá enviar sugerencias sobre el sistema. Las sugerencias Spam, de haberlas, deberán ser tratadas con una estrategia anti-spam.

\section*{Riesgos}

%Template
%\begin{riesgo}{SITUACION}{CONSECUENCIA}
%    \contexto{Texto de contexto}
%    \probabilidad{Baja-Media-Alta}
%    \impacto{Crítico-Medio-Bajo}
%    \exposicion{Alta-Media-Baja}
%    \mitigacion{Texto mitigacion}
%    \contingencia{Texto contingencia}
%\end{riesgo}

\begin{riesgo}{el/los cluster/s armados pueden no tener suficiente poder de computo}{los tiempos de respuesta y el conjunto de datos analizados no sean los esperados.}
    \contexto{Se espera manejar grandes volúmenes datos y de peticiones por lo tanto necesita alto poder de procesamiento y una gran cantidad de espacio para almacenarse.}
    \probabilidad{Media}
    \impacto{Crítico}
    \exposicion{Alta}
    \mitigacion{Diseñar arquitectura que escale horizontalmente y sea f\'acil agregar capacidad de procesamiento y de datos.}
    \contingencia{Disminuir la cantidad de información que se puede recibir y enviar TPA a los usuarios. Analizar otros esquemas arquitectónicos, como un esquema descentralizado distribuido en regiones.}
\end{riesgo}

\begin{riesgo}{los fondos sean insuficientes durante el contrato con Spam-Buster}{haya perdida de funcionalidad control de spam}
    \contexto{Dado el flujo en la economía argentina y los cambios en las modalidades de los proveedores existe el riesgo de no poder continuar con la contratación con Spam-Buster antes de haber terminado la funcionalidad que suplanta el servicio del proveedor.}
    \probabilidad{Baja}
    \impacto{Crítico}
    \exposicion{Alta}
    \mitigacion{Ser ceñudo en la gestión de los requerimientos de análisis de spam para que dichos CU no se retrasen.}
    \contingencia{Plantear clausula de cesantía de contrato para que sea desfavorable para Spam-Buster desligarse del proyecto en un lapso no menor a los 60 días.}
\end{riesgo}

\begin{riesgo}{surja un retraso en la implementación de reporte de ofertas dudosas}{se falte a la necesidades del sector Defensa al Consumidor}
    \contexto{La gente asociada al sector Defensa al Consumidor precisa este reporte para poder tomar las deciciones adecuadas, sin este el proyecto carece del aval institucional que prové DC.}
    \probabilidad{Baja}
    \impacto{Medio}
    \exposicion{Baja}
    \mitigacion{Priorizar estos casos de uso para garantizar la implementación de estos.}
    \contingencia{Proveer feedback inmediato al sector.}
\end{riesgo}

\begin{riesgo}{se genere una UI compleja}{se pierdan usuarios}
    \contexto{La aplicación TPA deberá tener alcance nacional y ser inclusiva para todos y todas. Las personas de edad avanzada o con capacidades especiales deberán poder utilizar las interfaces de usuario de manera intuitiva.}
    \probabilidad{Media}
    \impacto{Bajo}
    \exposicion{Baja}
    \mitigacion{Buscar especialistas en usabilidad para generar interfaces aptas.}
    \contingencia{No hacer nada, el porcentaje de estos usuarios es menor en comparación a los usuarios activos.}
\end{riesgo}

\begin{riesgo}{ataquen con una Negación de Servicio a la aplicación}{haya cientos de miles de usuarios damnificados}
    \contexto{Dada la puja política y económica, la aplicación podría afectar los intereses de ciertos grupos hegemónicos. Es esperable que se generen ataques de DoS.}
    \probabilidad{Media}
    \impacto{Crítico}
    \exposicion{Alta}
    \mitigacion{Generar una aplicación distribuida con buena distribución de carga para poder soportar alta carga de datos.}
    \contingencia{Minimizar el tiempo de reinicialización de los servicios de TPA.}
\end{riesgo}

\begin{riesgo}{oobo de información de usuarios}{falta a la privacidad definida en el EULA}
    \contexto{Dada la exposición de los usuarios y de sus preferencias de consumo, dicha información es suceptible a su comercialización.}
    \probabilidad{Alta}
    \impacto{Medio}
    \exposicion{Alta}
    \mitigacion{Anonimizar la información enviada por los usuarios para que otros usuarios vean inaccesible dicha información.}
    \contingencia{Cerrar las consultas de información y dejar solo accesible las consultas de ofertas generales.}
\end{riesgo}


\section*{Plan de Proyecto}

Considerando las posturas de los stakeholders con los que se tuvo contacto, se decidi\'o dar prioridad a la usabilidad, rendimiento, e integrabilidad y extensibilidad. Para esta segunda parte se cuenta con una dedicaci\'on \textit{full-time} de los tres integrantes del equipo.

\subsubsection{Iteraci\'on 1 - Elaboraci\'on (2 semanas / 240 horas)}	
	
	\begin{itemize}
		  \item Definici\'on de arquitectura
		  \item CU-01: Autentic\'andose al sistema
		  \item CU-02: Cargando oferta a trav\'es de la Web-API
		  \item CU-03: Consultando oferta a trav\'es de API-Internet
		  \item CU-04: Consultando oferta a trav\'es de p\'agina web TPA
		  \item CU-05: Enviando sugerencia sobre el sistema
	\end{itemize}

\textbf{Detalle de la iteración}

	Los casos de uso incluidos en esta primera iteraci\'on conforman un subconjunto m\'inimo que nos permite tener un recorrido completo de la aplicación. Esto \'ultimo implica tambi\'en que varios de \'estos tiene un alto impacto en la arquitectura, teniendo que tomar decisiones al respecto de la autenticaci\'on, implementaci\'on del servicio Web-API, integraci\'on del sitio Web-TPA con el servicio API, modelo de persistencia de entidades y resoluci\'on de b\'usquedas.
	
	La decisión de generar un recorrido completo también tiene en cuenta los riesgos relevados, dado que los riesgos de mayor exposición están relacionado con cuestiones de performance y arquitectura. Esto junto con la búsqueda de que haya una  adopción temprana del sistema por parte de los usuarios, permitirá evaluar tempranamente el funcionamiento del sistema para corregir en una fase temprana cualquier inconveniente de arquitectura que pudiera surgir y no hubiese sido relevado.
	
	Desde el punto de la usabilidad, incluimos el env\'io de sugerencia para tener un feedback temprano del usuario. Desde el punto de vista del rendimiento, apuntamos a utilizar una interfaz simple, que minimice el intercambio de datos por consulta. Por \'ultimo, desde el punto de vista de la integrabilidad y extensibilidad, decidimos implementar todos las funcionalidades en el servicio Web-API, reutilizándolo desde el sitio Web-TPA.
	
\subsubsection{Iteraci\'on 2 - Elaboraci\'on (2 semanas / 240 horas)}
	
	\begin{itemize}
	  \item CU-06: Configurando oferta sugerida (24hs)
	  \item CU-07: Buscando información de oferta sugerida (32hs)
	  \item CU-08: Generando reporte de ofertas dudosas con Spam-Buster (64hs)
	  \item CU-21: Configurando sistema de ofertas (56hs)
	  \item CU-10: Cargando/Consultando oferta por P\'agina Web/Red Social [Twitter] (64hs)
	\end{itemize}

\subsubsection{Iteraci\'on 3 - Elaboraci\'on (2 semanas / 240 horas)}
	
	\begin{itemize}
		\item CU-10: Cargando/Consultando oferta por P\'agina Web/Red Social [cont.] (64hs)
		\item CU-09: Invalidando oferta (40hs)
		\item CU-11: Generando reporte de ofertas dudosas con M\'odulo Propio (64hs)
		\item CU-12: Registrar informacion de uso del sistema (24)
		\item CU-13: Asignando confiabilidad a oferta (48)
	\end{itemize}

\subsubsection{Iteraci\'on 4 - Construcci\'on (4 semanas / 480 horas)}
	
	Las siguientes tareas se desarrollan para finalizar con el deplyment del sistema
	
	\begin{itemize}
	  	  \item CU-14: Asignando confiabilidad a usuario (40)
		  \item CU-10: Cargando/Consultando oferta por P\'agina Web/Red Social [cont.] (40hs)
		  \item CU-15: Comparando SpamBuster con M\'odulo de ofertas dudosas (160hs)
		  \item CU-16: Configurando sistema de confianza personal (136hs)
		  \item CU-17: Mostrando mapa oferta (104hs)
	\end{itemize}

\subsubsection{Iteraci\'on 5 - Transici\'on (2 semanas / 240 horas)}
	
	\begin{itemize}
		  \item CU-10: Cargando/Consultando oferta por P\'agina Web/Red Social [cont.] (60hs)
		  \item CU-18: Analizando web en busca de ofertas (60hs)
		  \item CU-18: Cargando/Consultando oferta por SMS (64hs)
		  \item CU-20: Consultando información estadística del sistema (56 hs)
	\end{itemize}

\subsection*{Gantt de la Primera Iteración}
\begin{figure}[hbtp]
\centering
\includegraphics[height=0.75\textheight,angle=90]{TP2Planificacion}
\caption{Diagrama de Gantt de tareas de la primera iteraci\'on, con divisi\'on de tareas y asignaci\'on de recursos}
\label{fig:gantt}
\end{figure}



\end{document}
