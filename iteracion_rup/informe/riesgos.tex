\section*{Riesgos}

%Template
%\begin{riesgo}{SITUACION}{CONSECUENCIA}
%    \contexto{Texto de contexto}
%    \probabilidad{Baja-Media-Alta}
%    \impacto{Crítico-Medio-Bajo}
%    \exposicion{Alta-Media-Baja}
%    \mitigacion{Texto mitigacion}
%    \contingencia{Texto contingencia}
%\end{riesgo}

\begin{riesgo}{cluster armado con insuficiente poder de computo}{fallar a nivel de datos de ofertas.}
    \contexto{Dado que se espera un volumen de tr\'afico alto es posible que el cluster no de abasto y el sistema no procese correctamente las ofertas, incumpliendo con los requerimientos de b\'usqueda o quedando fuera de servicio.}
    \probabilidad{Media}
    \impacto{Crítico}
    \exposicion{Alta}
    \mitigacion{Diseñar arquitectura que escale horizontalmente y sea f\'acil agregar capacidad de procesamiento}
    \contingencia{Detectar puntos cr\'iticos de infraestructura para poder agregar nuevas instancias de procesamiento.}
\end{riesgo}

\begin{riesgo}{los fondos sean insuficientes durante el contrato con Spam-Buster}{haya perdida de funcionalidad control de spam}
    \contexto{Dado el flujo en la economía argentina y los cambios en las modalidades de los proveedores existe el riesgo de no poder continuar con la contratación con Spam-Buster antes de haber terminado la funcionalidad que suplanta el servicio del proveedor.}
    \probabilidad{Baja}
    \impacto{Crítico}
    \exposicion{Alta}
    \mitigacion{Ser ceñudo en la gestión de los requerimientos de análisis de spam para que dichos CU no se retrasen.}
    \contingencia{Plantear clausula de cesantía de contrato para que sea desfavorable para Spam-Buster desligarse del proyecto en un lapso no menor a los 60 días.}
\end{riesgo}



