\section*{Riesgos}

%Template
%\begin{riesgo}{SITUACION}{CONSECUENCIA}
%    \contexto{Texto de contexto}
%    \probabilidad{Baja-Media-Alta}
%    \impacto{Crítico-Medio-Bajo}
%    \exposicion{Alta-Media-Baja}
%    \mitigacion{Texto mitigacion}
%    \contingencia{Texto contingencia}
%\end{riesgo}

\begin{riesgo}{el/los cluster/s armados pueden no tener suficiente poder de computo}{los tiempos de respuesta y el conjunto de datos analizados no sean los esperados.}
    \contexto{Se espera manejar grandes volúmenes datos y de peticiones por lo tanto necesita alto poder de procesamiento y una gran cantidad de espacio para almacenarse.}
    \probabilidad{Media}
    \impacto{Crítico}
    \exposicion{Alta}
    \mitigacion{Diseñar arquitectura que escale horizontalmente y sea f\'acil agregar capacidad de procesamiento y de datos.}
    \contingencia{Disminuir la cantidad de información que se puede recibir y enviar TPA a los usuarios. Analizar la distribución del sistema en nodos que abarquen las consultas de determinadas regiones.}
\end{riesgo}

\begin{riesgo}{la arquitectura se define con computadores de bajo costo}{se caíga la base de datos con la información de las ofertas}
    \contexto{La caída de la infraestructura de repositorios de ofertas generaría una caida del servicio completo.}
    \probabilidad{Baja}
    \impacto{Crítico}
    \exposicion{Alta}
    \mitigacion{Información guardada con chequeos de integridad que permitan una fácil recuperación (RAID). Redundancia de la base de datos con información de la ofertas.}
    \contingencia{Recuperar la información utilizando chequeo de integridad de datos. Levantar sistemas de base de datos de backup.}
\end{riesgo}

\begin{riesgo}{el sistema se espera sea usado masivamente}{se genere mucha carga en el uso del sistema que produzca lentitud de respuesta}
    \contexto{Dado que el sistema TPA tendrá impacto a nivel nacional y regional, es esperable una gran carga de consultas.}
    \probabilidad{Alta}
    \impacto{Crítico}
    \exposicion{Alta}
    \mitigacion{Generar una arquitectura que permita la alta disponibilidad del servicio utilizando \textbf{cachés} distribuidos para minimizar la carga en cada servidor. Armar casos de tests específicos para detectar problemas de disponibilidad en el sistema.}
    \contingencia{Levantar clusters adicionales, en caso de no ser posible evaluar la contratación de clusters externos.}
\end{riesgo}

\begin{riesgo}{se quiere que el sistema sea usado por la mayor cantidad de gente y en la mayor cantidad de plataformas}{no sea homogeneo entre las distintas plataformas y fácil de usar para todos los usuarios}
    \contexto{La aplicación TPA deberá tener alcance nacional y ser inclusiva para todos y todas. Las personas de edad avanzada o con capacidades especiales deberán poder utilizar las interfaces de usuario de manera intuitiva.}
    \probabilidad{Bajo}
    \impacto{Bajo}
    \exposicion{Baja}
    \mitigacion{Buscar especialistas en usabilidad para generar interfaces aptas.}
    \contingencia{No hacer nada, el porcentaje de estos usuarios es menor en comparación a los usuarios activos.}
\end{riesgo}

\begin{riesgo}{los servicios del sistema se proveerán a través de Internet}{recibir un ataque de Denegación de Servicio (DoS y DDoS) no siendo posible brindar los servicios}
    \contexto{Existen grupos de activistas que con distintas motivaciones atacan sitios mediante ataques de denegación de servicio para hacer llegar su mensaje. Estos ataques son faciles de generar y ejecutables en todo tipo de computadoras, incluso celulares o tablets, por lo que pueden ser facilmente escables por un grupo coordinado.}
    \probabilidad{Media}
    \impacto{Crítico}
    \exposicion{Alta}
    \mitigacion{Generar una aplicación distribuida con buena distribución de carga para poder soportar alta carga de datos. Organizar periodicamente simulaciones de este tipo de ataques para probar el sistema. }
    \contingencia{Minimizar el tiempo de reinicialización de los servicios de TPA. Analizar la distribución del sistema en nodos que abarquen las consultas de determinadas regiones. Evaluar la contratación de sistema contra este tipo de ataques.}
\end{riesgo}

\begin{riesgo}{se almacena información de los usuarios en sistema}{esa información sea robada}
    \contexto{Dada la exposición de los usuarios así como otra información que se requiera que sea relevante al sistema, ej: configuración de confianza, hábitos de utilización, dicha información es suceptible de ser comercializada.}
    \probabilidad{Alta}
    \impacto{Medio}
    \exposicion{Alta}
    \mitigacion{Anonimizar la información de los usuarios utilizada internamente. Aislar los datos sensibles de los usuarios de otros partes del sistema, encriptar, restringir y auditar su utilización. Realizar simulaciones de ataques para encontrar vulnerabilidades.}
    \contingencia{Cerrar las consultas de información y dejar solo accesible las consultas de ofertas generales.}
\end{riesgo}

\begin{riesgo}{dada la inmensa cantidad de intereses económicos involucrados}{se encuentre un alto porcentaje de ofertas dudosas}
    \contexto{Las ofertas dudosas son generadas por generadores de mercado que buscan manipular el sistema en pos de un beneficio particular.}
    \probabilidad{Bajo}
    \impacto{Medio}
    \exposicion{Media}
    \mitigacion{Mitigaremos este riesgo en principio utilizando los servicios de SpamBuster por su afamada excelencia en la detección de spam. Desarrollaremos una estrategia propia para poder desvincularnos del proveedor una vez terminada la implementación.}
    \contingencia{Habilitaremos listas de productos verificados por agentes de TPA que verifiquen personalmente las ofertas.}
\end{riesgo}

\begin{riesgo}{dada la inmensa cantidad de intereses económicos involucrados}{se encuentre un alto porcentaje de usuarios dudosos}
    \contexto{Los usuarios dudosos son generadores de mercado que buscan manipular el sistema en pos de un beneficio particular.}
    \probabilidad{Bajo}
    \impacto{Medio}
    \exposicion{Media}
    \mitigacion{Generaremos estratégias de validación de usuarios para minimizar la falsificación de identidad.}
    \contingencia{Habilitaremos listas de usuarios verificados por agentes de TPA.}
\end{riesgo}


\begin{riesgo}{dado el contexto inflacionario y de polarización política}{información no confiada por los usuarios}
    \contexto{La poca credibilidad en los índices inflacionarios podría inducir a la falta de confianza de los usuarios a los resultados de búsqueda.}
    \probabilidad{Baja}
    \impacto{Bajo}
    \exposicion{Baja}
    \mitigacion{Generaremos reportes de confianza para entender las necesidades de los usuarios y mejorar la confiabilidad de las ofertas sugeridas.}
    \contingencia{Generaremos una campaña mediática para generar confianza en los usuarios. Además mostraremos en el mapa los locales cercanos a su hogar, locales conocidos por los usuarios.}
\end{riesgo}

\begin{riesgo}{No utilización del sistema por usuarios pagos}{perdida de entrada de capitales}
    \contexto{El proyecto depende mucho de la redituabilidad de la plataforma. De no obtenerse sustentabilidad economica el proyecto podría caerse.}
    \probabilidad{Baja}
    \impacto{Crítico}
    \exposicion{Media}
    \mitigacion{Generaremos beneficios para los usuarios pagos garantizando ranking de las ofertas publicadas por ellos.}
    \contingencia{Agregaremos publicadad en el producto.}
\end{riesgo}


\begin{riesgo}{integración con servicios heterogeneos es compleja}{no descubre correctamente las ofertas}
    \contexto{Múltiples servicios integrados y alta complejidad en la detección de una oferta en lenguaje natural.}
    \probabilidad{Media}
    \impacto{Medio}
    \exposicion{Media}
    \mitigacion{Implementaremos estratégias de entrenamiento de la plataforma y generaremos múltiples casos de test para verificar la identificación de ofertas. También utilizaremos una estrategia de supervisión de la identifación de ofertas por un operario manual hasta y tanto la plataforma tenga un corpus de conocimiento adecuado para la interpretación de cualquier tipo de texto.}
    \contingencia{Contrataremos personal que identifique las ofertas que no se están encontrando para poder generar nuevas estrategias de detección.}
\end{riesgo}

\begin{riesgo}{la complejidad de desarrollo del proyecto es alta se precisa personal altamente calificado}{se generará un producto de baja calidad}
    \contexto{La complejidad, dada por los sistemas de asignación de confiabilidad, la gran interacción con sistemas heterogeneos, la necesidad de alta disponibilidad, etc. hace que el personal técnico que desarrollará la plataforma necesite estar altamente calificado.}
    \probabilidad{Alta}
    \impacto{Crítico}
    \exposicion{Alta}
    \mitigacion{Contratar personal altamente calificado y armar workshops para compartir el conocimiento de las tecnologías utilizadas en el proyecto hacia los menos calificados. Haciendo que todos los técnicos involucrados mantengan la calidad del producto.}
    \contingencia{Contratar un servicio de consultoría de software que permita resolver los conflictos téncnicos que pudieran darse.}
\end{riesgo}

\subsection*{Reveer}

Estos riesgos no se leen como \textit{DADO QUE condicion EXISTE EL RIESGO QUE evento}.
O no quedan claras las mitigaciones y/o contigencias.

\begin{riesgo}{los fondos sean insuficientes durante el contrato con Spam-Buster}{haya perdida de funcionalidad control de spam}
    \contexto{Dado el flujo en la economía argentina y los cambios en las modalidades de los proveedores existe el riesgo de no poder continuar con la contratación con Spam-Buster antes de haber terminado la funcionalidad que suplanta el servicio del proveedor.}
    \probabilidad{Baja}
    \impacto{Crítico}
    \exposicion{Alta}
    \mitigacion{Ser ceñudo en la gestión de los requerimientos de análisis de spam para que dichos CU no se retrasen.}
    \contingencia{Plantear clausula de cesantía de contrato para que sea desfavorable para Spam-Buster desligarse del proyecto en un lapso no menor a los 60 días.}
\end{riesgo}

\begin{riesgo}{surja un retraso en la implementación de reporte de ofertas dudosas}{se falte a la necesidades del sector Defensa al Consumidor}
    \contexto{La gente asociada al sector Defensa al Consumidor precisa este reporte para poder tomar las deciciones adecuadas, sin este el proyecto carece del aval institucional que prové DC.}
    \probabilidad{Baja}
    \impacto{Medio}
    \exposicion{Baja}
    \mitigacion{Priorizar estos casos de uso para garantizar la implementación de estos.}
    \contingencia{Proveer feedback inmediato al sector.}
\end{riesgo}
