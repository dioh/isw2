\documentclass[a4paper,8pt]{article}
\usepackage[utf8x]{inputenc}
\usepackage{listings}
%opening
\title{Ingeniería de Software II\\ \textbf{Sistema ``El precio justo''}}
\author{\textbf{Grupo 6}\\ 1º Cuatrimestre 2013} 
\date{}


\begin{document}

\maketitle
\vspace{10cm}
\begin{center}

\begin{tabular}{|c|c|c|}
\hline
\hline
\textbf{LU}&\textbf{Nombre}&\textbf{email}\\
\hline
& &\\
\hline
&&\\
\hline
&&\\
\hline
836/02&Paula Verghelet&pverghelet@gmail.com\\
\hline
\hline
\end{tabular}
\end{center}
\newpage

\section{Parte I}
\subsection{Descripción del problema, temas de discusión y decisiones tomadas}
\newpage
\subsection{Product Backlog}


\begin{tabular}{|p{1cm}|p{10cm}|p{1cm}|p{1cm}|}
\hline
\hline
\textbf{ID}&\textbf{Descripción}&\textbf{Est}&\textbf{Value}\\
\hline
\hline
US44 &	Como usuario quiero buscar por zona para realizar las compras cercanas a un lugar de mi elección &	8.0 	 &\\
\hline
%US45 &	Como usuario quiero buscar un producto en un rango de precios para poder seleccionar productos de acuerdo a lo que estoy dispuesto a pagar por el mismo 	&3.0  &\\	
\hline
%US47 &	Como usuario quiero minimizar el costo total de los productos buscados para realizar un ahorro en la compra total &	5.0 	& \\
\hline
%US48 &	Como usuario quiero minimizar el recorrido para minimizar el tiempo utilizado en la compra 	&13.0 	 &\\
\hline
%US49 &	Como usuario quiero buscar puntos de venta de un producto por cercaní­a a un lugar para poder elegir los más cercanos a dicho lugar &	5.0 	& \\
\hline
%US50 &	Como usuario quiero buscar puntos de venta de un producto por mejor precio para disminuir el costo de la compra 	&3.0 	& \\
\hline
%US51 &	Como usuario quiero ver todos los productos, con sus precios, asociados a un punto de venta para saber que otros productos se venden a buen precio en dicho punto de venta &	5.0 	 &\\
\hline
%US52 &	Como usuario quiero ver los resultados de una búsqueda en un mapa para ubicarme más facilmente en relación al lugar en el que me encuentro &	21.0&  \\
\hline	
%US55 &	Como administrador quiero administrar productos soportados para poder agregar quitar o editar los productos indexados &	5.0 	 &\\
\hline
%US56 &	Como administrador quiero habilitar sinónimos de los producots soportados para proveer mejores búsquedas a los usuarios 	&2.0 & \\
\hline	
%US57 &	Como sistema quiero permitir multiples tipos de unidades de los productos para unificar los precios de los productos sin importar la unidad 	&2.0 &	 \\
\hline
%US60 &	Como sistema quiero identificar productos por su nombre comercial para brindar información más precisa y/o no descartar twits que pueden ser útiles &	5.0  \\	
\hline
%US61 &	Como sistema quiero identificar la dirección en un tweet válido para poder indicar dónde se vende un producto a determinado precio &	2.0 	& \\
\hline
%US63 &	Como sistema quiero identificar direcciones como cruce de calles para poder suministrar información sobre la localización de venta del producto 	&13.0 	 &\\
\hline
%US64 &	Como sistema quiero identificar direcciones como calle más entre calles para poder suministrar información sobre la localización de venta del producto& 	5.0 & \\
\hline	
%US65 &	Como sistema quiero identificar lugares por nombre del local y barrio/sucursal para poder suministrar información sobre la localización de venta del producto &	3.0 	& \\
\hline
%US66 &	Como sistema quiero identificar la geolocalizació de un tweet para poder suministrar información sobre la localización de venta del producto 	&3.0 &	 \\
%US69 &	Como sistema quiero poder identificar productos escritos con faltas de ortografía para no descartar innecesariamente twits que pueden ser útiles 	&8.0 	& \\
\hline
%US81 &	Como sistema quiero poder identificar palabras que son similares (permutación válida) de los nombres de los productos para poder relevar una mayor cantidad de tweets &	8.0 	& \\
\hline
%US80 &	Como sistema quiero poder identificar en los tweets sinónimos de los productos para poder relevar mayor cantidad de información &	2.0 	& \\
\hline
%US107 &	Como sistema quiero poder identificar como válidos los tweets con formato $\#$PrecioJusto para poder procesarlos posteriormente &	2.0 &	 \\
\hline
%US53 &	[FUERA DEL ALCANCE DEL TP] Investigar beneficios e inconvenientes de que el producto tenga usuarios &	0.0 &	 \\
\hline
%US59 &	[FUERA DEL ALCANCE DEL TP] Investigar beneficios e inconvenientes de identificar subproductos como productos diferenciados &	0.0 	& \\
\hline
%US54 &	[FUERA DEL ALCANCE DEL TP] Investigar beneficios e inconvenientes de que el producto mantenga estadísticas de uso 	&0.0 	& \\
\hline
%US70 &	[FUERA DEL ALCANCE DEL TP] Investigar beneficios e inconvenientes de que el producto mantenga un sistema de veracidad/ranking de la información &	0.0 	& \\
%\hline
%US71 &	[FUERA DEL ALCANCE DEL TP] Como sistema quiero mostrar la información con mayor veracidad primero &&\\
\hline
\hline
\end{tabular}
\newpage

\subsection{Sprint Backlog}


\begin{tabular}{|p{1cm}|p{10cm}|p{1cm}|p{1cm}|}
\hline
\hline
\textbf{ID}&\textbf{Descripción}&\textbf{Est}&\textbf{Value}\\
\hline
\hline
US93&Como usuario quiero buscar por producto para poder elegir el lugar/precio que mejor se ajuste a mi necesidad.& 8& 7\\
\hline
\hline
\multicolumn{4}{|p{13cm}|}{ \textbf{Criterios de Aceptación}} \\
\hline
\hline
\multicolumn{4}{|p{13cm}|}{1. El usuario puede seleccionar un producto habilitado.}\\
\multicolumn{4}{|p{13cm}|}{2.  El usuario puede iniciar una búsqueda para este producto.}\\
\multicolumn{4}{|p{13cm}|}{3. El usuario puede ver los resultados para el producto buscado, con información de lugar y precio.}\\
\hline
\hline
\end{tabular}
\newpage
\end{document}
